\section{توضیح تمرین}
هدف شما در این تمرین پیاده سازی پروتکل DHCP در یک شبکه‌ی درختی است. تمرکز اصلی تمرین در لایه‌ی دو است و شما باید سعی کنید بسته‌ها را به درستی مسیریابی کنید و به مقصد برسانید.


دو نوع گره در شبکه وجود دارند که هر کدام کارهای مخصوص به خود را انجام می‌دهند. شما باید دستورات گفته شده برای هر کدام از این انواع را پیاده سازی کنید. برای داوری و اجرای برنامه‌ها شما نیاز به اتصال به شبکه پرتو دارید.

ساختار تمرین به این گونه است که شما باید در نقش هر کدام از کارگزار‌ها و کارخواه‌ها بسته‌ها را دریافت کنید، در صورت نیاز ارسال کنید و دستوراتی از صفحه کلید بگیرید و با توجه به دستورات کارهایی انجام دهید.

 نقش کارخواه این‌گونه است که این گره‌ها در شبکه حضور دارند و نیازمند دریافت آی‌پی هستند و کارگزاران باید برای هر کارخواهی که درخواست آی‌پی ارسال کرده است، آی‌پی جدیدی در نظر بگیرند. در این میان شما باید بسته ها را مسیریابی کنید و به مقصد برسانید. همچنین ممکن است آی‌پی‌ها منقضی شوند و یا کارخواهی درخواست آی‌پی جدیدی داشته باشد. پس به طور کلی شما باید همه این فرآیندها را پیاده سازی کنید.

\notice{
در طول تمرین می‌توانید فرض کنید در هر مرحله حداکثر یکی از فرآیندها در حال اجراست و تا تمام نشود، دستور بعدی وارد نمی‌شود. یعنی به طور مثال هنگامی که کارخواهی درخواست آدرس جدید می‌دهد، تا این بسته در کل شبکه نچرخیده باشد و تمام پیشنهاد‌های کارگزاران را ندیده باشد، در هیچ کارخواه دیگری دستوری وارد نمی‌شود. اما پس از آن ممکن است قبل از اینکه پیشنهادی را قبول کند، در کارخواه دیگری دستور درخواست آدرس جدید وارد شود.
}

\subsection{انواع بسته‌ها}
تمامی بسته‌هایی که در این تمرین تولید و بین گره‌ها جابه‌جا می‌شوند، ساختار زیر را دارند و شما موظفید تمام این قسمت‌ها را پر کنید، سپس بسته خود را ارسال کنید:

\begin{table}[htb]
\centering
  \begin{latin}
    \begin{tabular}{|c|c|}
      \hline
      
      \textcolor{purple}{Ethernet} & \textcolor{green}{Data}  \\
      \hline
      14 Byte & 11 Byte \\
      \hline
    \end{tabular}
  \end{latin}
\caption{ساختار بسته‌ها}
\end{table}

\subsubsection{
\textcolor{purple}{\lr{Ethernet}}:
}
آدرس مبدأ را آدرسی که در Interface ارسالی شما نوشته شده است بگذارید، آدرس مقصد را Broadcast و Type را برابر 0
(0x0000)
 قرار دهید.
\LTRfootnote{\url{https://en.wikipedia.org/wiki/Ethernet_frame\#Ethernet_II}}

\subsubsection{
\textcolor{green}{Data}:
}
با توجه به نوع بسته، محتویات بسته متفاوت خواهد بود. در ادامه جزییات این قسمت به صورت جدول آمده است. دقت کنید که همیشه قسمت MAC را برابر با مک آدرس واسط شماره ۰کارخواه بگذارید و هیچ‌گاه در قسمت IP و یا MAC اطلاعاتی از کارگزار قرار ندهید.

\begin{table}[htb]
\centering
  \begin{latin}
    \begin{tabular}{|c|c|c|c|c|}
     \hline
      Name & Sender & Data Type(1) & Mac(6) & IP(4) \\
      \hline
      DHCPDISCOVER& Client & 0 & Mac & 0 \\
      \hline
      DHCPOFFER & Server & 1 & Mac & Offer IP \\
      \hline
      DHCPREQUEST & Client & 2 & Mac & Offer IP \\
      \hline
      DHCPACK & Server & 3 & Mac & Offer IP \\
      \hline
      DHCPRELEASE & Client & 4 & Mac & Release IP \\
      \hline
      DHCPTIMEOUT & Server & 5 & Mac & Release IP \\
      \hline
      Request Extend & Client & 6 & Mac & Extend IP \\
      \hline
      Response Extend  & Server & 7 & Mac & New IP \\
      \hline
    \end{tabular}
  \end{latin}
\caption{انواع بسته‌ها}
\end{table}
