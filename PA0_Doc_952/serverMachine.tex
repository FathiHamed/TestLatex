\subsection{کارگزار DHCP}
شما در نقش کارگزار باید به درخواست‌های کارخواه‌ها پاسخ دهید و آدرس‌هایی که در اختیار دارید را به درستی مدیریت کنید. هر کارگزار یک
\lr{IP Pool}
دارد که مجموعه آدرس‌هایی است که در اختیار دارد. هر بار که یک IP را اختصاص می‌دهید باید از Pool خود خارج کنید تا به گره دیگری نسبت ندهید. شرح وظایف این نوع گره‌ها در ادامه آمده است.

\begin{itemize}
\item
مسیریابی بسته‌ها

در این نوع از گره‌ها در صورتی که بسته‌ای به دست شما رسید، نیازی نیست آن را به اطرافیان خود انتقال دهید. از این جهت این گره‌ها در درخت به نوعی مانند برگ در هر طرف یال خود هستند. همچنین بسته‌ای که به عنوان پاسخ بسته ورودی قرار است ارسال شود، تنها روی همان واسطی که بسته ورودی آمده، ارسال می‌شود.

\item
دریافت دستور
\begin{flushleft}
\code{add pool $IP$/$M$}
\end{flushleft}

این دستور به معنای افزودن محدوده جدیدی به مجموعه  آی‌پی‌های این کارگزار است. این محدوده به این صورت است که
$IP$
مقدار 
IP
است و 
$M$
مقدار
mask
آن، و شما باید کل آدرس‌های این بازه را (شامل آدرس اول و آخر) به مجموعه خود اضافه کنید. 

برای مثال، در صورتی که دستور
\code{add pool 192.168.1.10/30}
وارد شد، شما باید آدرس‌های،
\lr{192.168.1.8}،
\lr{192.168.1.9}،
\lr{192.168.1.10}،
\lr{192.168.1.11}
را به مجموعه اضافه کنید.

\notice{
تضمین می‌شود که $M$ در تمام تمرین تست‌ها بین ۲۵ تا ۳۲ باشد.
}


\notice{
تضمین می‌شود هیچ‌گاه دو کارگزار، محدوده مشترکی نداشته باشند. اما برای یک کارگزار ممکن است دو بازه مشترک داده شوند، شما باید اجتماع این بازه‌ها را بگیرید.
}
\item
دریافت بسته
\lr{DHCPDISCOVER}

پس از دریافت این بسته، شما باید اولین آی‌پی موجود خود را (کوچکترین) برای پیشنهاد ارسال کنید. (در قالب یک بسته‌ی
\lr{DHCPOFFER}
) و آن را از pool خود خارج کنید. دقت کنید که بسته را تنها روی واسطی می‌فرستید که بسته را از آن دریافت کرده‌اید. همچنین عبارت
\begin{flushleft}
\code{offer $IP$ to $MAC$}
\end{flushleft}
را چاپ کنید، که $IP$ مقدار IP  پیشنهادی و  $MAC$ آدرس MAC درخواست کننده است.

\item
دریافت بسته
\lr{DHCPREQUEST}

در صورت دریافت این بسته، اگر این IP متعلق به شما بود، باید بسته‌ای از نوع 
\lr{DHCPACK}
در پاسخ ارسال کنید و فرض کنید از این به بعد این IP اختصاص به این MAC آدرس دارد و عبارت
\begin{flushleft}
\code{assign $IP$ to $MAC$ for 10}
\end{flushleft}
را چاپ کنید. که $IP$ آدرس اختصاص داده شده و $MAC$ آدرس مک درخواست کننده است.

\notice{
از الآن تا قبل از ۱۰ واحد زمانی آینده، این قرارداد اعتبار دارد و اگر زمان را ۱۰ واحد به جلو ببریم، دیگر این قرارداد اعتباری ندارد.
}
\item
دریافت بسته 
\lr{DHCPACK}

در صورتی که این بسته را دریافت کردید، به این معناست که گره مورد نظر، درخواست شما را رد کرده است. در نتیجه باید IP که به او پیشنهاد داده بودید را دوباره به Pool خود بازگردانید و عبارت
\begin{flushleft}
\code{$IP$ back to pool}
\end{flushleft}
را چاپ کنید. که $IP$ مقدار آی‌پی بازگردانده شده است.

\item
دریافت بسته
\lr{DHCPRELEASE}

اگر این IP را شما اختصاص داده بودید، حال باید آن را به Pool بازگردانید و فرض کنید به گره‌ای اختصاص ندارد.

\item
دریافت دستور
\begin{flushleft}
\code{add time $t$}
\end{flushleft}

باید زمان کارگزار را $t$ واحد جلو ببرید. در صورتی که در این زمان، IP ای منقضی شده بود، آن را به Pool بازگردانید و سپس بسته 
\lr{DHCP Timeout}
را برای اطلاع کارخواه بفرستید.
ترتیب ارسال بسته‌ها باید به ترتیب زمان انقضا باشد. اگر دو زمان انقضا باهم برابر بودند، به هر ترتیب دلخواهی می‌توانید آن دو را بفرستید.
\item
دریافت بسته 
\lr{Request Extend}

در پاسخ، همیشه افزایش زمان را تایید می‌کنید؛ اما ابتدا آدرس قبلی را با اضافه کردن به Pool بی اعتبار می‌کنید و پس از آن، کوچکترین آدرسی که در Pool موجود است را (ممکن است دوباره همین آدرس باشد) با زمانی معادل با زمان باقیمانده از آدرس قبلی، به‌علاوه 10 به کارخواه اختصاص می‌دهید. بسته را ارسال می‌کنید و در خروجی عبارت:
\begin{flushleft}
\code{assign $IP$ to $MAC$ for $t$}
\end{flushleft}
را چاپ می‌کنید. در این عبارت آدرس جدید، مک کارخواه و زمان جمع زده شده را چاپ می‌کنید.

\item
دریافت دستور
\begin{flushleft}
\code{print pool}
\end{flushleft}

باید کل Pool را به ترتیب از کوچک به بزرگ چاپ کنید. دقت کنید که آی‌پی‌‌هایی که  در حال حاضر اختصاص  یا OFFER داده‌اید را نباید چاپ کنید.

\end{itemize}
