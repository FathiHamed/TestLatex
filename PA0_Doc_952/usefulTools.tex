\newpage
\section
{
\textbf{ {\Large 
بخش دوم: چند ابزار مفید
}}}
در اسن بخش با چند ابزار مفید آشنا می شوید که معمولا در جعبه ابزار هر سامانه حمله کننده ای معمولا یافت می شود. از میان این ابزار یادگیری git و make الزامی هستند، زیرا بدون کسب دانش کافی از این ابزار قادر نخواهید بود کد خود را کامپایل نموده و سپس ارسال کنید. سایر ابزار یا در جهت رفع باگ به کار میروند یا در جهت چندوظیفگی\LTRfootnote{multitasking} با کارای بیشتر مورد استفاده قرار می گیرند.
\notice{
تمامی این ابزارها بر روی ماشین مجازی نصب گردیده اند.
}

\subsection{Git}
یک برنامه version control است که به کمک آن میتوانید روند کدها را دنبال کنید. GitHub یکی از سامانه های تحت وب است که امکان hosting کدهای شما را برایتان فراهم می کند. درواقع این سامانه فضای تعاملی و اشتراک گذاری کدها را فراهم ساخته است.
اگر چه تا این مرحله شما تنظیمات مقدماتی را انجام داده اید، اما تسلط شما به قابلیتهای git میتواند در طول این درس به کمک شما بیاید، خصوصا هنگامی که با هم گروهی هایتان تمرین را انجام می دهید.
اگر مایلید که سطح دانش خودتان را نسبت به گیت افزایش دهید میتوانید اینجا را مشاهده کنید.

\subsection{make}
make ابزاری است که به صورت خودکار برنامه های اجرایی و کتابخانه ها را از کد منبع تولید می کند و این کار را به کمک خواندن فایل Makefile انجام می دهد. Makefile تعیین می کند که چگونه به برنامه هدف دسترسی پیدا کند. به این صورت که لیست تمامی وابستگی ها\LTRfootnote{dependency} را در آن قرار می هدید و make با پیمیایش آنها برنامه اجرایی شما را تولید می کند.
متاسفانه make syntax بسیار بدی دارد که اگر به صورت درست از آنها استفاده نکنید برای فهم آنها دچار مشکل خواهید شد.
بنابراین توصیه می شود از اینجا آموزش لازم را فرا بگیرید.
فعلا ما از ساده ترین فرم make که نیازی به Makefile ندارد استفاده می کنیم. بنابراین با اجرای دستور زیر میتوانید به راحتی کامپایل کرده به wc.c متصل شوید.\begin{flushleft}
\code{make wc}
\end{flushleft}
این دستور یک فایل اجرایی ایجاد کرد که شما میتوانید اجرایش کنید. حال دستور زیر را اجرا کنید:
\begin{flushleft}
\code{./wc wc.c}
\end{flushleft}
تفاوت دستور بالا با دستوری که در ادامه می آید چیست؟ (راهنمایی: ابتدا دستور “which wc” را اجرا کنید.)
\begin{flushleft}
\code{wc wc.c}
\end{flushleft}

تمرین اول شما این خواهد بود که wc.c را به گونه ای تغییر دهید که تعداد کلمات را با توجه به ویژگیهای “man wc” پیاده سازی کند و در آن تنها نیاز دارید که از یک فایل ورودی پشتیبانی کنید(یا STDIN اگر هیچ ویژگی تعیین نشده بود).
توجه کنید که wc در OS X کاملا متفاوت از Ubuntu عمل می کند، بنابراین انتظار می رود که رفتار آن را در Ubuntu دنبال کنید.

\subsection{gdb}
دیباگ کردن برنامه های با زبان C بسیار سخت است اما خوشبختانه gdb امکان دیباگ کردن آسان را فراهم نموده است. اگر شما برنامه هایتان را با پرچم\LTRfootnote{flag} خاص -g کامپایل کنید و برنامه را داخل gdb اجرا کنید علاوه بر stack trace میتوانید متغیرها را inspect کنید، غییر دهید، کد را متوقف کنید و ... .
gdb ساده امکانات کمی دارد، به همین دلیل cgdb روی ماشین مجازی برایتان نصب شده است که یک سری قابلیتها از قبیل syntax highlighting را داراست. در cgdb میتوانید با استفاده از i و ESC بین پنجره\LTRfootnote{pane} های بالایی و پایینی switch کنید.
همچنین gdb میتواند پردازه های جدید را شروع کند و به پردازه های درحال اجرا ملحق کند.
برای یادگیری gdb پیشنهاد می کنیم اینجا را مشاهده کنید. البته مستند اصلی gdb نیز علی رغم طولانی بدن مفید است.
برای یادگیری بیشتر تلاش کنید با wcتان کار کنید. با پرچم\LTRfootnote{flag} -g برنامه تان را کامپایل کنید. برنامه را از طریق gdb شروع کنید کنید و یک نقطه وقفه\LTRfootnote{break point} بر سر main بگذارید. سپس برنامه را تا آنجا اجرا کنید و دستورهای مختلف را تمرین کنید. بفهمید که چگونه آرگومان های خط فرمان را میتوان pass داد. متغیرهای محلی اضافه کنید و مقادیر\LTRfootnote{value} نظیرشان را پیدا کنید. step ، next و break را فرابگیرید.

\subsection{tmux}
یک multiplexer مربوط به terminal است که چندین tab مربوط به terminalرا شبیه سازی می کند اما آنها را در یک session از terminal نمایش می دهد. البته این چند tab را هنگام ssh به ماشین مجازی حفظ می کند.
شما میتوانید یک session جدید را با دستور \begin{flushleft}
\code{tmux new -s <session_name>}
\end{flushleft} ایجاد کنید. سپس میتوانید با فشار دادن \begin{flushleft}
\code{ctrl-b + c}
\end{flushleft} یک پنجره\LTRfootnote{window} جدید ایجاد کنید و با فشار دادن \begin{flushleft}
\code{ctrl-b + n}
\end{flushleft} به پنجره nام پرش کنید.
اگر \begin{flushleft}
\code{ctrl-b + d} را فشار دهید از session مربوط به tmux جدا می شوید درحالیکه کماکان درحال اجراست و هر برنامه ای درون آن نیز درحال اجراست. برای آنکه session خود را ادامه دهید میتوانید از \begin{flushleft}
\code{tmux attach -t <session_name>}
\end{flushleft} استفاده کنید.
برای شروع به یادگیری tmux میتوانید اینجا را مشاهده کنید.

\subsection{vim}
یک ویرایشگر متن زیبا برای استفاده درون terminal است. بسیاری نیز emacs را بر vim ترجیح می دهند شمااما آنچه که اهمیت دارد آن است که در یک ویرایشگر به تسلط برسید تا بتوانید در نوشتن کدها از آن بهره ببرید. برای تسلط در vim میتوانید اینجا را مطالعه کنید.

\subsection{ctags}
از آنجاکه تعداد خط کد بسیاری را خواهید خواند این ابزار در استفاده از وقتتان صرفه جویی می کند و سبب می گردد که بین قطعات مختلف کد navigate کنید.
دستورالعمل نصب این ابزار را برای vim از اینجا و برای sublime از اینجا مشاهده کنید. البته از ویرایشگرهای دیگر نیز پشتیبانی می کند که در این صورت میبایست دستورالعمل مرتبط را جستجو کنید.