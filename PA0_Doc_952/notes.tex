\clearpage
{\LARGE  \textbf{نکات ضروری}}
\begin{itemize}
\item
به علت اینکه نمره‌ی تمرین به صورت خودکار داده می‌شود، ساختار پیام‌های
مطرح‌ شده باید دقیقاً به صورتی باشد که در مستند توضیح داده شده است. 
\item
نقشه‌ای که برای ارزیابی استفاده می‌شود با نقشه تست که در اختیار شما قرار گرفته متفاوت خواهد بود. 
\item
داوری خودکار به صورت کامل در اختیار شما قرار داده می‌شود و می‌توانید نمره خود را ببینید. اما ملاک ارزیابی نمره‌ای است که کد ارسالی شما روی کارگزار  خواهد گرفت. اگر موارد گفته شده را رعایت کرده باشید، نمره شما نباید تغییری داشته باشد.
\item
به دلیل مشکلات اینترنتی بهتر است داوری را هنگامی که به شبکه‌ی دانشگاه متصل هستید انجام دهید.
\item
در صورتی‌که هر مشکل یا پرسشی داشتید که فکر می‌کنید پاسخ آن برای همه مفید خواهد بود،
	آن را به گروه اینترنتی درس ارسال کنید.
\item
از فرستادن جواب تمرین به گروه اینترنتی درس خودداری کنید.
\item
 تمام برنامه‌ی شما باید توسط خود شما نوشته شده باشد. فرستادن کل یا قسمتی
	از برنامه‌تان برای افراد دیگر، یا استفاده از کل یا قسمتی از برنامه‌ی فرد دیگری، حتی با
	ذکر منبع، تقلب محسوب می‌شود.
\item
 پس از اتمام کارتان لازم است با اجرای دستور
\code{make archive}
فایل زیپی شامل تمام فایل‌هایی که برای اجرا شدن کد شما نیاز است بسازید.(این دستور فایل info.sh شما را درون زیپ قرار نمی‌دهد زیرا نیازی به این فایل نیست!) در صورتی که از کلاس‌ها و فایل‌های اضافه شده خودتان استفاده می‌کنید، سعی کنید در پوشه گفته شده باشد. در هر صورت فایل آرشیو شما باید قابلیت کامپایل/اجرا شدن را به روش سیستمی داشته باشد، در غیر اینصورت نمره شما صفر خواهد شد.
\item
نسخه نهایی تمرین خود را به
\href{http://quera.ir/}{
وب‌سایت کوئرا
}
ارسال نمایید.
\end{itemize}
