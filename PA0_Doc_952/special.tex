\subsection{موارد خاص}
در این قسمت حالت‌های خاصی از مسئله که ممکن است پیش بیاید و در حالت‌های مسئله مطرح نشده است بررسی می‌شود. نکته بسیار مهم این است که کد شما به هیچ وجه نباید در زمان تست از کار بیافتد، زیرا ممکن است نمره برخی قسمت‌ها را به صورت کامل از دست بدهید.
در زیر حالت‌های مهم که در تست‌ها باید رعایت شوند آمده است و کافی است همین حالت‌ها را بررسی کنید:
\begin{itemize}
\item
دور ریختن و چاپ عبارت 
\begin{flushleft}
\code{invalid packet, dropped}
\end{flushleft}
برای بسته‌هایی که 
\code{Data Type}
آن‌ها جزء موارد گفته شده نیست.

\item
برای گره‌های کارخواه و کارگزار در صورتی که دستور وارد شده غلط باشد، باید عبارت:
\begin{flushleft}
\code{invalid command}
\end{flushleft}
را چاپ کنید. و منتظر دستورات بعدی باشید.

\end{itemize}
