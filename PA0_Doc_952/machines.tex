\subsection{کارگزار}
در این تمرین تنها یک کارگزار داریم که در آدرس عمومی
\code{1.1.1.1}
قرار دارد و به درگاه 
\code{1234}
گوش می‌کند. شما به عنوان کارخواه باید بسته‌های خود را به این آدرس و درگاه بفرستید. همچنین مبدا تمام بسته‌هایی که از طریق کارگزار ایجاد و ارسال می‌شوند، باید با این آدرس و درگاه پر شوند.
کار‌های کارگزار به ترتیب در زیر آمده‌اند:

\begin{itemize}
\item
دریافت بسته نوع 0 از کار‌خواهان و سپس اختصاص $ID$ جدید به آن‌ها و چاپ پیام:
\begin{flushleft}
\code{new id $ID$ assigned to $IP$:$Port$}
\end{flushleft}
که
\lr{
$IP$:$Port$
}
مشخصات عمومی کارخواه فرستنده پیام است. در انتها بسته‌ نوع ۰ ای شامل این $ID$ تولید کرده و به کارخواه فرستنده، باز می‌گرداند.
\notice{
$ID$
های تولید شده به ترتیب از ۱ شروع می‌شوند و هربار یک واحد افزایش می‌یابند. بنابر این شماره اولین درخواست کننده عدد ۱، دومین درخواست کننده عدد ۲ و الی آخر است.
}
\item
هنگامی که کارگزار بسته نوع 1 را از همسایه‌ها می‌گیرد، ابتدا پیام زیر را چاپ می‌کند.
\begin{flushleft}
\code{$ID_A$ wants info of node $ID_B$}
\end{flushleft}
که 
\code{$ID_A$}
و
\code{$ID_B$}
به ترتیب ID مبدا(کسی که بسته از آن آمده) و مقصد(کسی که قرار است مبدا به آن متصل شود) است.

سپس باید یک بسته نوع 1 شامل اطلاعات مقصد($B$) تولید کند و به آدرس عمومی گره مبدا($A$) می‌فرستد.
\end{itemize}

\subsection{کارگزار NAT}
وظیفه کارگزار NAT در این شبکه به این شکل است که تمام واسط‌های خود بجز واسط ۰ را زیر شبکه‌ی خود می‌بیند و از طریق واسط ۰ به دنیای بیرون متصل است. در نتیجه آدرس و درگاه بسته‌های ورودی و خروجی به زیر شبکه‌هایش را عوض می‌کند.
وظایفی که برعهده دارد به شرح زیر است:
\begin{itemize}
\item
تغییر مبدا بسته‌هایی که از زیر شبکه وارد می‌شوند با این الگوریتم:
\subitem
اگر تا بحال بسته‌ای با این آدرس و درگاه مبدا وارد نشده بود، یک آدرس و درگاه عمومی جدید به این اختصاص بده و این‌ مقادیر را جایگزین کن(ترجمه کن).
\subitem
در صورتی که از قبل این آدرس و درگاه ترجمه شده بودند، این بار نیز از همان ترجمه قبلی استفاده کن.

و اگر بسته‌ای از بیرون شبکه وارد می‌شد، تنها در صورتی ترجمه و مسیریابی می‌شود که از قبل بسته‌ای در جهت عکس برای این گره فرستاده شده باشد. بنابر‌این گره‌های بیرونی نمی‌توانند هیچ بسته‌ای به داخل زیر شبکه NAT بفرستند مگر اینکه از قبل بسته‌ای برای آن‌ها فرستاده شده باشد.

الگوریتم اختصاص آدرس و درگاه عمومی به این صورت است:
با شروع از درگاه 2000 و اولین آدرس بعد از آدرس واسط شماره ۰، هربار درگاه را ۱۰۰ واحد اضافه می‌کنید، تا به عدد 2200 برسید، در این صورت شما باید آدرستان را یک واحد افزایش دهید و درگاه را دوباره از 2000 شروع کنید. برای واضح تر شدن قضیه به مثال زیر توجه کنید:

فرض کنید آدرس واسط ۰ شما برابر با
\code{1.2.3.4}
  است. بنابراین آدرس‌های عمومی‌ای که تولید می‌کنید به ترتیب برابر با:
\subitem
\code{1.2.3.5:2000}
\subitem
\code{1.2.3.5:2100}
\subitem
\code{1.2.3.6:2000}
\subitem
\code{1.2.3.6:2100}
\subitem
...
می باشد.
\item
دریافت دستور
\begin{flushleft}
\code{block port range $Port_{min}$ $Port_{max}$}
\end{flushleft}
که تنظیمات NAT را انجام می‌دهد. این تنظیمات تعیین می‌کند که محدوده درگاه مبدا بسته‌هایی که از زیر شبکه این NAT می‌آیند، در چه بازه‌ای باشد. با هربار دریافت این دستور، شما باید بازه‌ای شامل خود اعداد گفته شده را مسدود کنید و بسته‌های این بازه را پس از فرستادن پیام
\code{drop}
، دور بریزید. ممکن است چندین بار این دستور با محدوده های مشترک و غیر مشترک وارد شود، شما باید اجتماع تمام این بازه‌ها را مسدود کنید.

\item
فرستادن بسته نوع 0 با پیام drop در صورتی که بسته نوع ۰ ای از زیر شبکه خود دریافت کردید، اما بسته در محدوده block قرار داشت.
\end{itemize}
